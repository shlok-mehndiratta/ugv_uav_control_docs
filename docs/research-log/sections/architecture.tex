\section{Implementation Architecture}

\subsection{ROS 2 Node Graph}

The system is organized as a modular ROS~2 computation graph integrating simulation, perception, planning, and control. The architecture follows a layered design that clearly separates responsibilities while maintaining deterministic data flow.

\subsubsection*{Simulation Layer}
The simulation layer provides the physical and sensory environment:
\begin{itemize}
\item Ground robot exposing a velocity command interface.
\item Aerial platform supplying an overhead visual perspective.
\item Ground plane environment used for path visualization.
\end{itemize}

\subsubsection*{Middleware Bridge}
A middleware bridge connects the simulator to the ROS~2 ecosystem:
\begin{itemize}
\item Translates simulator-specific messages into ROS~2-compatible formats.
\item Converts velocity commands into stamped messages.
\item Forwards aerial camera data to downstream perception nodes.
\end{itemize}

\subsubsection*{Perception and Planning}
This layer extracts navigational intent from sensor data:
\begin{itemize}
\item Overhead image processing extracts path geometry.
\item Waypoints are generated and published for downstream control.
\end{itemize}

\subsubsection*{Control Layer}
The control layer executes motion commands based on waypoint input:
\begin{itemize}
\item Multiple interchangeable controllers (Stanley, Pure Pursuit, MPC).
\item Consumes waypoint messages and outputs velocity commands.
\item Commands are routed back to the simulator through the middleware bridge.
\end{itemize}

This layered organization enforces separation of concerns while maintaining a clear and deterministic flow of information.

\subsection{Message Type Design Decision}

\paragraph{Issue Encountered}
A mismatch was observed between the velocity command message type expected by the simulation bridge and the type originally published by the controllers.

\paragraph{Resolution}
All controller outputs were standardized to publish stamped velocity messages, including both a timestamp and a reference frame identifier.

\paragraph{Rationale}
The use of stamped messages provides several advantages:
\begin{itemize}
\item Enables future latency compensation and time-aligned control.
\item Supports reliable data association for logging and debugging.
\item Ensures compliance with recommended ROS~2 communication practices.
\end{itemize}

This design choice improves robustness and extensibility without increasing controller complexity.

\subsection{Launch System Architecture}

The system launch configuration is designed to maximize flexibility and runtime configurability. Its primary responsibilities include:

\begin{itemize}
\item \textbf{Dynamic Model Loading}:  
Simulation models are loaded at runtime, programmatically modified to substitute visual elements (e.g., track textures), and instantiated into the environment.

\item \textbf{Controller Selection}:  
The active control strategy is selected via a runtime argument, enabling seamless switching between Stanley, Pure Pursuit, and MPC without code modification.

\item \textbf{Environment Configuration}:  
Simulation resource paths are extended at launch time to ensure that custom models and assets are discoverable by the simulator.
\end{itemize}

This launch architecture decouples configuration from implementation, enabling rapid experimentation and reproducibility.

