\section{Problem Formulation}

\subsection{Original Problem Statement}

The fundamental problem addressed by this project is vision-based autonomous
path following for a differential-drive ground robot (TurtleBot3 Burger)
using an overhead camera perspective.

The system is required to:
\begin{itemize}
\item Localize the robot without relying on GPS or SLAM, using fiducial marker detection.
\item Extract navigable path geometry from camera imagery.
\item Generate waypoints that guide the robot along the path centerline.
\item Execute control commands that minimize tracking error while respecting kinematic constraints.
\end{itemize}

\subsection{Constraint Hierarchy}

\begin{table}[h]
\centering
\caption{Constraint hierarchy governing the TurtleBot3 platform}
\label{tab:constraint_hierarchy}
\begin{tabular}{lll}
\toprule
\textbf{Constraint Type} & \textbf{Mathematical Form} & \textbf{Physical Interpretation} \\
\midrule
Non-holonomic &
$\dot{x}\sin\theta - \dot{y}\cos\theta = 0$ &
No lateral motion \\
Velocity bounds &
$0 \le v \le 0.22$ &
TurtleBot3 motor limits \\
Angular rate bounds &
$|\omega| \le \omega_{\max}$ &
Actuator limits \\
Wheel slip &
Implicit friction model &
$\mu = 1000$ assumption \\
\bottomrule
\end{tabular}
\end{table}

\subsection{Problem Evolution}

The scope of the problem evolved through the following stages:
\begin{itemize}
\item \textbf{Phase 1}: Static overhead camera with direct path following.
\item \textbf{Phase 2}: UAV-mounted camera for future mobile observation.
\item \textbf{Phase 3}: Multi-controller architecture with modular selection.
\end{itemize}

