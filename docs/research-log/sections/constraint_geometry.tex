\section{Constraint Geometry}

\subsection{Non-Holonomic Constraint}

The mobile base is subject to a non-holonomic motion constraint, meaning it cannot generate lateral velocity. This restriction is captured by the standard unicycle kinematic model:
\[
\dot{x} = v \cos\theta,
\qquad
\dot{y} = v \sin\theta.
\]

There is no independent control over $\dot{x}$ and $\dot{y}$; translational motion is constrained to lie strictly along the instantaneous heading direction.

\paragraph{Consequence}
Path planning and control algorithms must explicitly account for a minimum achievable turning radius. Given a bounded angular velocity, the minimum turning radius is
\[
r_{\min}
=
\frac{v}{\omega_{\max}}
=
\frac{0.22}{2.84}
\approx 0.077~\mathrm{m}.
\]

This constraint limits the maximum curvature of feasible trajectories and rules out instantaneous lateral motion.

\subsection{Actuator Constraints}

Actuation limits are imposed by the differential-drive motor model, which enforces saturation on wheel speeds. Control inputs are expressed as linear and angular velocity commands.

\paragraph{Command Structure}
The accepted command interface consists of:
\begin{itemize}
\item \textbf{Linear velocity}: Forward translational speed, subject to motor saturation limits.
\item \textbf{Angular velocity}: Yaw rate, internally converted into differential wheel velocities.
\end{itemize}

These actuator constraints bound the feasible control inputs and must be respected by higher-level planners and controllers to ensure physically realizable motion.

