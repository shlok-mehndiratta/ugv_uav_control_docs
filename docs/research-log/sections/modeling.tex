\section{Modeling Assumptions}

\subsection{Robot Kinematic Model}

The TurtleBot3 Burger is modeled as a planar unicycle system with state
\[
\mathbf{x} =
\begin{bmatrix}
x & y & \theta
\end{bmatrix}^{T}
\in \mathbb{R}^2 \times S^1,
\]
where $(x,y)$ denotes the robot position in the world frame and $\theta$ is the heading (yaw) angle.

\paragraph{Robot Parameters}
The physical parameters used in the model are:
\begin{itemize}
\item Wheel separation: $L = 0.160~\mathrm{m}$
\item Wheel radius: $r = 0.033~\mathrm{m}$
\item Mass: $m = 0.825~\mathrm{kg}$
\item Maximum linear acceleration: $a_{\max} = 1.0~\mathrm{m/s^2}$
\end{itemize}

\paragraph{Differential-Drive to Unicycle Mapping}
The relationship between differential wheel velocities and unicycle control inputs is given by
\[
v = \frac{r(\omega_R + \omega_L)}{2}, \qquad
\omega = \frac{r(\omega_R - \omega_L)}{L},
\]
where $\omega_R$ and $\omega_L$ denote the angular velocities of the right and left wheels, respectively.

\subsection{Camera Model}

The overhead camera is modeled using an ideal pinhole projection model. The assumed camera parameters are:
\begin{itemize}
\item Horizontal field of view: $\phi = 1.047~\mathrm{rad}$ ($60^\circ$)
\item Camera height above the ground plane: $h = 10.0~\mathrm{m}$
\item Image resolution: $1920 \times 1080~\mathrm{px}$
\end{itemize}

\paragraph{World-to-Image Scaling}
The physical width of the observable ground region is
\[
w_{\text{world}} = 2h \tan\!\left(\frac{\phi}{2}\right).
\]

Using the image width $w_{\text{image}}$, the pixel-to-meter conversion factor is
\[
s = \frac{w_{\text{world}}}{w_{\text{image}}}
\approx 0.00601~\mathrm{m/px}.
\]

\paragraph{Critical Assumption}
The camera optical axis is assumed to be perfectly vertical. Any deviation from this alignment introduces perspective and radial distortions, which are not corrected in the current implementation.

\subsection{ArUco Marker Localization}

Robot pose estimation is performed using an ArUco fiducial marker (ID~55, DICT\_6X6\_250). The marker is mounted $0.2~\mathrm{m}$ above the robot base on a $0.5~\mathrm{m} \times 0.5~\mathrm{m}$ planar plate.

\begin{itemize}
\item \textbf{Heading Estimation}:  
The robot heading is estimated from the orientation of the detected marker as
\[
\theta_{\text{robot}} =
\operatorname{atan2}
\bigl(y_{\text{top}} - y_{\text{center}},\;
      x_{\text{top}} - x_{\text{center}}\bigr),
\]
where $(x_{\text{top}}, y_{\text{top}})$ denotes the midpoint of the marker’s top edge and $(x_{\text{center}}, y_{\text{center}})$ is the marker center.

\item \textbf{Limitation}:  
Heading estimation accuracy degrades as the marker moves farther from the camera center due to perspective distortion. No geometric correction or compensation is applied.
\end{itemize}

