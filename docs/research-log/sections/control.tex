\section{Control Strategy Evolution}

\subsection{Pure Pursuit Controller}

\paragraph{Mathematical Foundation}
Pure Pursuit is a geometric path-following controller that computes a circular arc connecting the robot to a look-ahead waypoint.

The curvature of the arc is given by
\[
\gamma = \frac{2y}{L_d^2},
\qquad
L_d = \sqrt{x^2 + y^2},
\]
where $L_d$ is the Euclidean distance to the waypoint and $y$ is the lateral offset in the robot frame.

The commanded angular velocity is
\[
\omega = v \cdot \gamma.
\]

\paragraph{Implementation}
In the implemented formulation, the curvature is computed as
\[
\gamma = \frac{2y}{d^2},
\]
and the angular velocity command is
\[
\omega = v_{\text{target}} \cdot \gamma.
\]

\paragraph{Design Rationale}
\begin{itemize}
\item Simpler than Stanley control, as it does not require an explicit heading error term.
\item More robust to noise in waypoint orientation.
\item Exhibits natural corner-cutting behavior, which is beneficial for speed optimization.
\end{itemize}

\paragraph{Limitation}
Pure Pursuit does not explicitly correct heading error; as a result, the robot may approach the path with a misaligned orientation.

\subsection{Stanley Controller}

\paragraph{Mathematical Foundation}
Derived from Stanford’s DARPA Grand Challenge entry, the Stanley controller minimizes cross-track error (CTE) and heading error simultaneously:
\[
\delta
=
\psi_e
+
\arctan\!\left(
\frac{k_{cte} \cdot e_{cte}}{1 + v}
\right).
\]

The terms are defined as:
\begin{itemize}
\item $\psi_e$: Heading error (target heading minus robot heading).
\item $e_{cte}$: Cross-track error (lateral deviation from the path).
\item $k_{cte} = 2.0$: Cross-track error gain.
\item $v$: Current forward velocity.
\end{itemize}

\paragraph{Implementation}
The cross-track error is computed as
\[
e_{cte}
=
y \cos(\theta_{\text{target}})
-
x \sin(\theta_{\text{target}}).
\]

The steering command is then
\[
\delta
=
\psi_e
+
\operatorname{atan2}(k_{cte} \cdot e_{cte},\, 1 + v),
\]
and the resulting angular velocity command is
\[
\omega
=
\frac{v \tan(\delta)}{0.2},
\]
where $0.2~\mathrm{m}$ is the approximate wheelbase.

\paragraph{Design Rationale}
\begin{itemize}
\item Guarantees convergence to the path for bounded initial CTE.
\item Exhibits natural velocity-dependent damping (higher speed implies weaker correction).
\item Well-documented and empirically reliable behavior.
\end{itemize}

\paragraph{Limitation}
The Stanley controller assumes knowledge of the path tangent; in the current implementation, this tangent is approximated using waypoint orientation.

\subsection{Model Predictive Control (MPC)}

\paragraph{Choice Justification}
Model Predictive Control was introduced to overcome the limitations of geometric controllers:
\begin{itemize}
\item Enables planning over a finite prediction horizon.
\item Explicitly enforces actuator constraints.
\item Optimizes the trade-off between tracking accuracy and control smoothness.
\end{itemize}

\subsubsection{State Augmentation}

The kinematic state is augmented to include actuator dynamics:
\[
\mathbf{x}_{\text{aug}} =
\begin{bmatrix}
x & y & \theta & v_{\text{act}} & \omega_{\text{act}}
\end{bmatrix}^{T}.
\]

\paragraph{Actuator Dynamics}
First-order actuator dynamics are modeled as
\[
\dot{v}_{\text{act}} = \alpha_v (v_{\text{cmd}} - v_{\text{act}}),
\qquad
\dot{\omega}_{\text{act}} = \alpha_\omega (\omega_{\text{cmd}} - \omega_{\text{act}}).
\]

The corresponding time constants are
\[
\tau_v = 0.5~\mathrm{s},
\qquad
\tau_\omega = 0.2~\mathrm{s},
\]
chosen empirically.

\subsubsection{Integration Method Evolution}

\paragraph{Initial Approach (Forward Euler)}
The initial discretization uses Forward Euler integration:
\[
\mathbf{x}_{k+1}
=
\mathbf{x}_k
+
f(\mathbf{x}_k, \mathbf{u}_k)\,dt.
\]

This approach introduces systematic bias at high angular velocities, causing curved trajectories to degenerate into straight-line chords.

\paragraph{Improved Approach (Exact Arc Integration)}
For $|\omega| \ge \epsilon$, with $\epsilon = 10^{-4}$, the exact integration of circular motion is used:
\[
x_{k+1}
=
x_k
+
\frac{v}{\omega}
\bigl[\sin(\theta_k + \omega dt) - \sin\theta_k\bigr],
\]
\[
y_{k+1}
=
y_k
-
\frac{v}{\omega}
\bigl[\cos(\theta_k + \omega dt) - \cos\theta_k\bigr],
\]
\[
\theta_{k+1}
=
\theta_k
+
\omega dt.
\]

For $|\omega| < \epsilon$, the straight-line approximation is applied:
\[
x_{k+1} = x_k + v \cos\theta_k\,dt,
\qquad
y_{k+1} = y_k + v \sin\theta_k\,dt.
\]

\subsubsection{Cost Function}

The MPC objective function is defined as
\[
J
=
\sum_{k=1}^{N}
\Bigl[
w_d \|\mathbf{p}_k - \mathbf{p}_{\text{target}}\|^2
+
w_h (\theta_k - \theta_{\text{target}})^2
+
w_r \|\mathbf{u}_k - \mathbf{u}_{k-1}\|^2
\Bigr].
\]

\begin{table}[h]
\centering
\caption{MPC cost function weights}
\label{tab:mpc_weights}
\begin{tabular}{l c l}
\toprule
\textbf{Weight} & \textbf{Value} & \textbf{Purpose} \\
\midrule
$w_d$ & 10.0 & Minimize distance to target \\
$w_h$ & 2.0 & Minimize heading error \\
$w_r$ & 5.0 & Penalize control rate (smoothness) \\
\bottomrule
\end{tabular}
\end{table}

The target heading is computed greedily as
\[
\theta_{\text{target}}
=
\operatorname{atan2}
(y_{\text{target}} - y,\; x_{\text{target}} - x).
\]

This formulation directly points toward the target; a path-tangent-based reference would provide improved anticipatory behavior.

\subsubsection{Optimization}

\paragraph{Solver}
Sequential Least-Squares Quadratic Programming (SLSQP) is used for optimization.

\paragraph{Decision Variables}
\[
\mathbf{u}
=
[v_0, \omega_0, v_1, \omega_1, \ldots, v_{N-1}, \omega_{N-1}]^{T}.
\]

\paragraph{Constraints}
\begin{itemize}
\item Velocity bounds: $0 \le v_k \le 0.22$.
\item Angular velocity bounds: $-2.0 \le \omega_k \le 2.0$.
\item Rate constraints enforced implicitly via a soft penalty in the cost function.
\end{itemize}

