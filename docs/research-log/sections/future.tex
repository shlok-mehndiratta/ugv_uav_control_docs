\section{Future Research Directions}

\subsection{Immediate Extensions (Engineering)}

Several near-term engineering extensions can significantly improve system robustness, performance, and deployability.

\begin{itemize}
\item \textbf{Warm-start optimization}:  
Initialize the optimizer using a shifted version of the previous solution to improve convergence speed and reduce computational overhead.

\item \textbf{Path spline fitting}:  
Fit a cubic spline to the extracted path centerline in order to ensure trajectory continuity and smooth curvature profiles.

\item \textbf{Odometry integration}:  
Fuse wheel odometry with visual marker-based localization to obtain more robust and drift-resistant state estimation.
\end{itemize}

\subsection{Theoretical Directions}

Beyond implementation improvements, several control-theoretic extensions can provide stronger analytical guarantees and cleaner problem formulations.

\begin{itemize}
\item \textbf{Frenet-frame MPC}:  
Reformulate the optimization problem in path-relative (Frenet) coordinates, enabling more interpretable cost terms and explicit separation of longitudinal and lateral errors.

\item \textbf{Terminal cost}:  
Introduce a terminal penalty $\phi(\mathbf{x}_N)$ to improve closed-loop stability and convergence guarantees.

\item \textbf{Explicit MPC}:  
Pre-compute the control law offline to enable deterministic real-time execution with bounded computational complexity.
\end{itemize}

\subsection{Learning-Based Extensions}

Learning-based approaches offer the potential to augment or partially replace model-based components in complex environments.

\begin{itemize}
\item \textbf{Cost function learning}:  
Apply inverse reinforcement learning to infer MPC cost weights from expert demonstrations.

\item \textbf{Path extraction}:  
Employ semantic segmentation networks to enable robust path perception in visually complex or unstructured environments.

\item \textbf{End-to-end control}:  
Use imitation learning to directly map sensory observations to control commands, bypassing explicit path extraction and geometric modeling.
\end{itemize}

\subsection{Multi-Agent Coordination}

Extending the framework to cooperative multi-agent scenarios opens several additional research directions.

\begin{itemize}
\item \textbf{UAV--UGV cooperative planning}:  
Leverage aerial sensing to provide long-horizon look-ahead information for ground vehicle navigation.

\item \textbf{Dynamic path modification}:  
Allow the aerial agent to modify visual cues or markers in real time to influence ground navigation behavior.

\item \textbf{Distributed MPC}:  
Perform coordinated trajectory optimization across multiple agents while respecting individual dynamics and communication constraints.
\end{itemize}

