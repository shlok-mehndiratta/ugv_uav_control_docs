\section{Numerical Methods Analysis}

\subsection{Discretization Error}

Forward Euler integration exhibits a local truncation error of order $O(dt^2)$ per integration step, which accumulates to a global error of order $O(dt)$ over a finite prediction horizon.

For a sampling period of $dt = 0.2~\mathrm{s}$ and a prediction horizon of $N = 5$, the following observations apply:
\begin{itemize}
\item The maximum trajectory drift over the horizon becomes significant for high-curvature paths.
\item Exact arc integration eliminates this discretization error for motion segments with constant $(v, \omega)$.
\end{itemize}

\subsection{Optimization Convergence}

The Sequential Least-Squares Quadratic Programming (SLSQP) algorithm does not guarantee convergence for non-convex optimization problems. In this implementation, the cost function is non-convex due to:
\begin{itemize}
\item Trigonometric terms appearing in the state propagation equations.
\item Angle wrapping effects in the heading error formulation.
\end{itemize}

\paragraph{Mitigation Strategy}
Convergence robustness is improved by initializing the optimizer using the solution from the previous control step. Additionally, the maximum number of solver iterations is capped at $20$ to bound computation time.

\subsection{Numerical Stability}

The exact arc integration formulation contains the term $\tfrac{v}{\omega}$, which becomes ill-conditioned as $\omega \rightarrow 0$.

\paragraph{Stability Handling}
To maintain numerical stability, the integration scheme switches to a straight-line approximation when
\[
|\omega| < 10^{-4}.
\]

This conditional handling prevents numerical singularities while preserving accuracy in the low-curvature limit.

