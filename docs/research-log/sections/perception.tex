\section{Perception Pipeline}

\subsection{Lane Segmentation via Floodfill}

The path extraction pipeline consists of the following stages:
\begin{itemize}
\item \textbf{Thresholding}: Binary intensity thresholding at a value of 100.
\item \textbf{Seed Point Search}: Probing candidate seed points within a radial range of 30--70~px and an angular window of $\pm 0.6~\mathrm{rad}$.
\item \textbf{Floodfill Propagation}: Connected-component extraction using a floodfill algorithm.
\item \textbf{Distance Transformation}: Euclidean distance transform
\(
\mathcal{D}: I \rightarrow \mathbb{R}^{+}.
\)
\end{itemize}

The path centerline is defined as the locus of points maximizing the distance transform:
\[
\mathbf{p}^{*}
=
\arg\max_{\mathbf{p} \in \mathcal{R}} \mathcal{D}(\mathbf{p}),
\]
where $\mathcal{R}$ denotes the set of pixels belonging to the extracted path region.

\subsection{Waypoint Generation}

Waypoints are generated in the robot body frame using a polar-to-Cartesian transformation. Given a target distance $d$ and relative bearing $\alpha$, the waypoint coordinates are
\[
x = d \cos\alpha, \qquad
y = d \sin\alpha.
\]

The relative bearing is computed as
\[
\alpha
=
-\left(
\theta_{\text{target,image}}
-
\theta_{\text{robot}}
\right).
\]

Expressing the distance explicitly in metric units yields
\[
x = d_{\text{meters}} \cos\alpha, \qquad
y = d_{\text{meters}} \sin\alpha.
\]

\paragraph{Coordinate Convention}
All waypoints are expressed in the robot body frame, with the following axis definitions:
\begin{itemize}
\item $+x$: Forward (aligned with the robot heading)
\item $+y$: Left (perpendicular to heading, right-hand rule)
\item Origin: Robot center
\end{itemize}

